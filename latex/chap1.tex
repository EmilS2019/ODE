\section{Simple early model}

\begin{flushleft}
    We want to analyze how a rainbowfish population shiftes inside
    of an aquarium over time. We wan't to know if it's feasable to
    sell 20 fishes per day and still retain a population. This is why
    we need to find the equalibrium points, were the fish population is
    stable.
\end{flushleft}


\begin{flushleft}
    For that we need to construct an Ordinary Differential Equation.
    The growth rate of the rainbowfish population is 70\% with
    a maximum aquarium capacity of 750 fishes. The death rate
    is 0.001 times the population, aka one fish lives for 1000
    days. But this is negligable and is removed. 20 rainbowdishes are
    bought every day, so that's included in the model.
\end{flushleft}

\begin{equation}
    \frac{dP}{dt} = 0.7P(t)(1-\frac{P(t)}{750})-20
\end{equation}

\begin{flushleft}

    The model was then solved numericaly in Python using
    eulers formula with $\Delta t=\frac{1}{16}$
    which gave the following results:

\end{flushleft}

\begin{center}
    \includegraphics[scale=0.4]{../figures/Figure_1.png}
\end{center}

\begin{flushleft}
    As we can see, the amount of rainbowfish approaches somewhere
    above 700.
    The exact amount is 720.2, but since you can't have
    fractional fish we can approximate it to 720. This is
    a stable equilibrium point as it increases from below
    and increases from above.
\end{flushleft}

\begin{flushleft}
    As a conclusion this model does work for one fish,
    however we want to introduce a second type of fish
    into the model. The grourami.
\end{flushleft}


