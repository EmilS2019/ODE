\section{Second model}

\subsection{Deriving a system of ODE's with gourami fish}
\begin{flushleft}
    The fish owner desperately wanted two types of fish species:
    rainbowfish and gourami. The current supply and budget allows for 20 rainbowfish and 5 gourami.
    To make sure that both fishes both had enough food and didn't go extinct a system of differential equations were derived.
    There is a 4\% chance that a gourami will kill a rainbowfish, or in math terms $-0.04PG$.
    Gouramis also don't survive on their own so they slowly die, in math terms $-0.25G$
\end{flushleft}

\begin{align*}[left = \empheqlbrace]
    \frac{dP}{dt}= 0.7P-0.007P^2-0.04PG \\
    \frac{dG}{dt} = -0.25G+0.008PG
\end{align*}

\subsection{Numerically solving the system in Python}
\begin{flushleft}
    Now doing the rest was easy. So easy in fact that I wanted to shoot my foot
    with a rocket launcher. The results are shown below.

\end{flushleft}
\begin{figure}[H]
    \includegraphics[scale=0.4]{../figures/Figure_2.png}
    \centering
    \caption{Number of rainbowfish(Blue) and gourami (Orange) over time. Note that the
        rainbowfish population is shifting slightly "behind" the gourami population}
    \label{two}
\end{figure}

\begin{figure}[H]
    \includegraphics[scale=0.4]{../figures/Figure_3.png}
    \centering
    \caption{The amount of Gouramis on the y-axis and the amount of
        rainbowfish on the x-axis.}
\end{figure}

\begin{flushleft}
    These models tells us that the fish seem to approach an equilibrium at roughly
    30 rainbowfishes and 12 gourami. The exact equilibrium point is when $\frac{dP}{dt}=0$ and $\frac{dG}{dt}=0$.
    This evaluates to the number of rainbowfish being 31.25 and the amount of gourami is 12.03125. Again, since you
    cannot have fractional fishes the equilibrium is approximated to be at 31 rainbowfish and 12 gourami.

\end{flushleft}

\begin{flushleft}
    Modelling a few more scenarios with different
    starting populations reveal that there is a stable equilibrium point
    as long as the amount of fish of one species is more than 0. The reason
    for doing this is for extra accuracy.
    If you plot fish populations against each other rather than over
    time as shown in figure \ref{vortex} it creates a "vortex shape"
    since the populations always approach the equilibrium value.
\end{flushleft}

\begin{figure}[H]
    \centering
    \includegraphics[scale=0.4]{../figures/Figure_4.png}
    \caption{Many different starting populations to illustrate a "vortex shape".
        Note that if you begin
        with 0 rainbowfish the gourami always drop to 0 and if you begin with 0
        gourami then the rainbowfish will approach 100}
    \label{vortex}
\end{figure}


